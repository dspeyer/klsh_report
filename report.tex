\documentclass[twoside,11pt]{homework}
\usepackage{lipsum}
\usepackage{mathrsfs, bbm, tikz-cd}


\coursename{COMS 6998-5 Fall 2017}
\studentname{Daniel Speyer and Geelon So}      % YOUR NAME GOES HERE
\studentmail{\{dls2192,geelon\}@columbia.edu}   % YOUR UNI GOES HERE
\homeworknumber{Final Report}               % THE HOMEWORK NUMBER GOES HERE
\collaborators{none}             % THE UNI'S OF STUDENTS YOU DISCUSSED WITH


\let\oldto\to
\renewcommand{\to}{\longrightarrow}

\begin{document}
\maketitle

\begin{abstract} Test
\end{abstract}

\section{Introduction}

%\subsection{LSH} % dspeyer

\section{Kernels} % geelon

One of the most natural ways to encode structure into an abstract set $\mathcal{X}$ is to provide a mapping  from that set to the reals, $\mathcal{X} \to \bbR$. Or more generally, a mapping to some structure-rich mathematical object $\mathcal{O}$. Indeed, at a high level, we can consider different mathematical objects as the `canonical' idealization of certain types of structure; then, certain functions from $\mathcal{X}$ to these objects allow us to `pull back' or instantiate that structure within our specific $\mathcal{X}$.

Of course, the structure we care about with respect to LSH is \emph{similarity}, one idealization of which is the \emph{(real) inner product}. Let us take a moment to reexamine the inner product on finite real vector spaces as a notion of similarity, before generalizing that notion using \emph{kernels} to infinite dimensional vector spaces and arbitrary sets.

\subsection{Finite-Dimensional Inner Product Spaces}
\begin{definition}\label{IP}
  Let $V$ be a finite vector space over $\bbR$. An \emph{inner product} on $V$ is a bilinear map $K : V \times V \to \bbR$
  that is \emph{symmetric} and \emph{positive-definite}. We call the pair $(V,K)$ an \emph{inner product space}.
\end{definition}
Symmetry and positive-definiteness are two of the most obvious requirements for any self-respecting function that calls itself a measure of similarity:
\begin{enumerate}
\item  the similarity between two points $x$ and $y$ better not depend on the order we present them to the similarity measure, so we require $K(x,y) = K(y,x)$,
\item a nonzero object should be similar to itself, so if $x \ne 0$, then $K(x,x) > 0$.
\end{enumerate}
These two intuitions of a `similarity measure' then precisely define the conditions of an inner product on $V$.\footnote{For now, we can just view bilinearity as a necessary condition to ensure that $K$ respects the real numbers as an algebraic object. However, on arbitrary sets, it is unclear what (bi)linearity should mean.} Allow us to be (perhaps overly) rigorous in the following discussion, for it is better to bore in the finite case than to confuse in the general case.

For concreteness, let $V = \bbR^n$ with a fixed basis. With any $x,y \in V$, denote by $\langle x,y\rangle$ the \emph{formal dot product} of $x$ and $y$ as an algebraic construction.\footnote{One point of confusion may be that $\langle \cdot,\cdot \rangle$ often denotes an inner product---usually unproblematic as the dot product is in fact the inner product \emph{induced by the choice of basis} on $V$. However, an abstract vector space has no canonical basis; thus, it has no canonical inner product. In particular, the inner product corresponding to the dot product here is in general unrelated to the inner product $K$. We will reconcile these two views at Equation~\ref{canonical-inner-prod}, which shows that every inner product corresponds to the dot product with respect to some basis. Conversely, as every dot product is an inner product, if $V$ is an abstract finite-dimensional real vector space, \emph{specifying an inner product on $V$ is equivalent to specifying a basis on $V$}.} Then, as $K$ is a bilinear form, there exists a matrix $\mathbf{K}$ where
\begin{equation}\label{kernel-history}
  K(x,y) = \langle x, \mathbf{K}y\rangle.
\end{equation}
The symmetry condition on the inner product $K$ forces the matrix $\mathbf{K}$ to be symmetric. The spectral theorem on symmetric matrices tell us that $V$ has an eigenbasis with respect to $\mathbf{K}$. Consider an eigenvalue $K v = \lambda v$. As we require $K$ to be positive-definite, we know that
\[K(v,v) = \langle v, \mathbf{K}v\rangle = \lambda \langle v,v\rangle> 0.\]
Since the dot product of a vector with itself in any basis is nonnegative, this implies that the eigenvalue $\lambda$ is positive; so in its eigenbasis, $\mathbf{K}$ is diagonal with positive terms on its diagonal---it follows that $K^{1/2}$ exists, and this allows us to write:
\[K(x,y) = \langle \mathbf{K}^{1/2} x, \mathbf{K}^{1/2}y\rangle.\]
We deduce that there exists a linear map $\Phi: V \to \bbR^n$ (spoiler: we call $\Phi$ a \emph{feature map}) where the inner product on $V$ directly corresponds to the usual dot product on $\bbR^n$. That is,
\begin{equation}\label{canonical-inner-prod}
  K(x,y) = \langle \Phi(x), \Phi(y)\rangle.
\end{equation}
In other (ridiculously abstract) words, Equation~\ref{canonical-inner-prod} precisely says:
\begin{proposition}\label{finite-IP}
  Let $V$ be an $n$-dimensional real vector space. If $K : V \times V \to \bbR$ is an inner product, then there exists a map $\Phi: V \to \bbR^n$ such that the following map commutes:
  \[\begin{tikzcd}
V \times V\ \  \arrow[dashed, rr, "\Phi \times \Phi"] \arrow[drr, "K"'] & & \bbR^n \times \bbR^n \ \arrow[d, "{\langle\cdot,\cdot \rangle}"] \\
& & \mathbb{R}
\end{tikzcd}
\]
  where $\langle \cdot, \cdot \rangle$ is the standard dot product on $\bbR^n$.
\end{proposition}
This is the main payoff to all this formality: we may view $(\bbR^n, \langle\cdot,\cdot\rangle)$ as the canonical $n$-dimensional real inner product space. So, no matter the which inner product space $(V,K)$ we want to study, we're in fact guaranteed the existence of an isomorphism $\Phi$ to the familiar inner product space $\bbR^n$, and we may freely interchange the two objects in our minds.

Now, let us extend this to notions of similarity on abstract sets.

\subsection{Kernels and Hilbert Spaces}
First, we will need to extend the `similarity measure' from vector spaces $V$ to abstract sets $\mathcal{X}$. In the former, notice that given a basis $\{v_1,\dotsc, v_n\}$, the inner product $K$ is fully determined by the values of $K(v_i,v_j)$ ranging over all $1\leq i,j\leq n$. In some sense, this says that there are only `$n$ ways' or `$n$ directions' in which objects of $V$ may be (dis)similar.

This suggests that we can generalize Definition~\ref{IP} because we didn't need all of $V$ to start with! Suppose someone secretly had an $n$-dimensional inner product space $(V,K)$, but just gave us $n$ linearly independent vectors, say $[n] := \{v_1,\dotsc, v_n\}$, and the corresponding restriction of the similarity measure $K: [n] \times [n] \to \bbR$.

Our view of $[n]$ would just be a collection of $n$ abstract objects, and $K$ just an $n\times n$ matrix over $\bbR$ (this is often called the \emph{Gram matrix}). Still, we would have produced the same feature map $\Phi_{[n]}: [n] \hookrightarrow \bbR^n$, albeit restricted to $[n]\subset V$. But in the previous analysis, since $V$ and $\bbR^n$ are isomorphic via $\Phi$, this implies that $\Phi_{[n]}$ actually \emph{recovers} $V$ by its identification with $\bbR^n$; it is as though we've discovered that $[n]$ actually lived inside the space $V$.

Very naturally, the generalization of $K$ from $[n]$ to $\mathcal{X}$ takes the view that $K$ is a similarity `matrix' on $\mathcal{X}$:
\begin{definition} Let $\mathcal{X}$ be a nonempty set. A \emph{kernel}\footnote{Unfortunately, the terminology \emph{kernel} can be terribly confusing. Within functional analysis, kernels are often interchangably called \emph{kernel maps}, \emph{kernel functions}, and \emph{positive-definite kernels}. Yet, each of these also take on other meanings depending on author and context. `Kernels' and `kernel maps' are almost always synonymous. Sometimes, `kernel functions' denote kernels where it is possible to write $K(x,y) = k(x-y)$, where subtraction must also be defined. And in many instances, when context is clear, the author cares only for positive-definite kernels, so omitting the specifier \emph{positive-definite}.

    Furthermore, \emph{kernels} used in this sense are unrelated to the algebraic kernel of a linear map. The terminology comes from the theory of integral operators. In particular, infinite-dimensional function spaces often have inner products of the form:
    \[\langle f, g\rangle = \int_{X \times X} K(x,y) f(x) g(y) \ \mathrm{d}x \mathrm{d}y,\]
    where $K$ performs the analogous role as $\langle x, \mathbf{K}y\rangle$ in Equation~\ref{kernel-history} and precisely the role we desire in the general case. These maps $K$ were historically called kernels.
  } is a map $K : \mathcal{X} \times \mathcal{X} \to \bbR$. We say that $K$ is \emph{positive-definite} if for any finite subsets $A$ of $\mathcal{X}$, the corresponding Gram matrix of $A$ is symmetric and positive-definite.
\end{definition}

















\subsection{Unusual Distance Metrics} % dspeyer

\section{KLSH}

\subsection{Getting a Normal Distribution} % dspeyer

\subsection{Approximating the Covariance} % geelon

\subsubsection{Interpretation as Projection} % geelon

\subsubsection{Points to be Dropped} % dspeyer

\section{Tweaking Parameters} % geelon

\subsection{Numbers of Points}

\subsection{Number of Eigenvectors}

\section{Data-Dependent KLSH} % dspeyer

\subsection{Data-Dependent LSH}

\subsubsection{Approximate Evenness}

\subsection{Smaller Caps}

\subsection{Making the Calculations}

\section{References}

\end{document}
