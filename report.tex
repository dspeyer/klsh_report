\documentclass[twoside,11pt]{homework}
\usepackage{lipsum}

\usepackage{mathrsfs, bbm, tikz-cd, multicol}


\coursename{COMS 6998-5 Fall 2017}
\studentname{Daniel Speyer and Geelon So}      % YOUR NAME GOES HERE
\studentmail{\{dls2192,geelon\}@columbia.edu}   % YOUR UNI GOES HERE

\homeworknumber{Final Report}               % THE HOMEWORK NUMBER GOES HERE
\title{Kernelized Locality Sensitive Hashing}
\collaborators{none}             % THE UNI'S OF STUDENTS YOU DISCUSSED WITH


%\let\oldto\to
%\renewcommand{\to}{\longrightarrow}

\begin{document}
\maketitle

\begin{abstract}
\end{abstract}

%\begin{multicols}{2}

\section{LSH} % dspeyer

Locality sensitive hashing is a general solution for searches in which
the desired key is near or nearest to the query, rather than equal.
It is not as efficient as tree-based solutions for $\bbR^n$ with small
$n$, but unlike those solutions, it maintains its performance in high
dimensions or in arbitrary metric spaces.

Most near search problems can be reduced to cr-near search.  Suppose
we have a database of points $x_1,x_2,...,x_n$ drawn from some
arbitrary metric space $\cX$ and some query $q$ also from $\cX$, plus
constants $c,r\in\bbR$.  A cr-near search seeks either an $x$ within
distance $cr$ of $q$ or a statement that there are none within $r$.
It is generally best to think of $r$ as part of the query and $c$ as
the acceptable level of error.

To use LSH for problems like this we will need a family of hashing
functions $h_1,h_2,...$ that map $\cX$ into some small finite space,
all following two key inequalities:

\begin{eqnarray*}
  ||x_1-x_2|| < r & \rightarrow & Pr[h(x_1)=h(x_2)] \geq p_1 \\
  ||x_1-x_2|| > cr & \rightarrow & Pr[h(x_1)=h(x_2)] \leq p_2 \\
\end{eqnarray*}

Note that we may require an arbitrary number of hashing functions,
which must be generated randomly.  It is this randomness that allows
the use of probabilities in the preceding inequalities.  (The
distribution of hashing functions is often written $\cH$, but we'll be
reserving that symbol for Hilbert spaces later.)

In the simplest example, $\cX$ is a high dimensional Hamming space and
the hash functions sample random columns.  This may provide useful
intuition for thinking about hash functions.

Once we have a source of functions and a gap between $p_1$ and $p_2$,
we can widen the gap by taking $n$ functions and concatenating their
outputs (getting $p_1^n$ and $p_2^n$).  We can then deal with too low
a $p_1^n$ by taking $m$ repetitions and searching all of them (getting
$mp_1^n$ and $mp_2^n$).  This solves the cr-near problem with
arbitrarily high probability, which can in turn solve other, similar
problems.

%TODO: Write about efficiency

\section{Kernels} % geelon

In this section, we will motivate and describe the theory of kernel methods. We often develop mathematics and algorithms for objects that explicitly encode a lot of structure. For example, LSH has been developed for Hamming spaces, finite-dimensional vector spaces, $n$-spheres, and so on. These objects are particularly well-suited for analysis, as the very way we \emph{represent} them directly showcases their \emph{properties}. Often, the harmony between representation and structure empowers analysis and enables elegance.

But we can't expect harmony always. Nonetheless, might we be able to extend these methods to objects that have implicit structure?

Kernel methods are one way of approaching such situations. Let $\cX$ be our object---just an abstract set. But perhaps it has implicit structure. How might we access this structure? Very naturally, we can encode this structure into a mapping from that abstract set to a structure-rich mathematical object.

We often do this without thinking, for example, assigning numbers to objects in order to compare them---a mapping of the form $\mathcal{X} \to \bbR$. Indeed, at a high level, we can consider different mathematical objects as the `canonical' idealization of certain types of structure; then, functions from $\mathcal{X}$ to these objects allow us to `pull back' or instantiate that structure within our specific $\mathcal{X}$.

Of course, the structure we care about with respect to LSH is \emph{similarity}, one idealization of which is the \emph{(real) inner product}. In fact, with kernel methods, we will be able to tightly bind any abstract object to an inner product space, provided the object has the right notion of similarity. This will provde us both new insight and new algorithms. Let us take a moment to reexamine the inner product on finite-dimensional real vector spaces, before generalizing to infinite-dimensional vector spaces and arbitrary sets.

\subsection{Finite-Dimensional Inner Product Spaces}
\begin{definition}\label{IP}
  Let $V$ be a (possibly infinite) vector space over $\bbR$. An \emph{inner product} on $V$ is a bilinear map $K : V \times V \to \bbR$
  that is \emph{symmetric} and \emph{positive-definite}. We call the pair $(V,K)$ an \emph{inner product space}.
\end{definition}
Symmetry and positive-definiteness are two of the most obvious requirements for any self-respecting function that calls itself a measure of similarity:
\begin{enumerate}
\item  the similarity between two points $x$ and $y$ better not depend on the order we specify them, so we require $K(x,y) = K(y,x)$,
\item a nonzero object should be similar to itself, so if $x \ne 0$, then $K(x,x) > 0$.
\end{enumerate}
These two simple intuitions of a `similarity measure' then precisely define the conditions of an inner product on $V$.\footnote{For now, we can just view bilinearity as a necessary condition to ensure that $K$ respects the real numbers as an algebraic object. So, in some sense, bilinearity is not intrinsic to $K$; rather, it is a condition \emph{induced} by the field $\bbR$. Later on, when we generalize, we won't have a notion of linearity on arbitrary sets $\mathcal{X}$; however, there is a natural `completion' of $\mathcal{X}$, a vector space $\mathcal{H}$. As before, the linear structure will be induced by $\bbR$.} Allow us to be (perhaps overly) rigorous in the following discussion, for it is better to bore in the finite case than to confuse in the general case.

For concreteness, let $V$ be an $n$-dimensional real vector space with a fixed basis. With any $x,y \in V$, denote by $\langle x,y\rangle$ the \emph{formal dot product} of $x$ and $y$ as an algebraic construction.\footnote{One point of confusion may be that $\langle \cdot,\cdot \rangle$ often denotes an inner product---usually unproblematic as the dot product is in fact the inner product \emph{induced by the choice of basis} on $V$. However, an abstract vector space has no canonical basis; thus, it has no canonical inner product. In particular, here, the inner product corresponding to the dot product is in general unrelated to the inner product $K$. We will reconcile these two views at Equation~\ref{canonical-inner-prod}, which shows that every inner product corresponds to the dot product in an appropriate basis. Conversely, as every dot product is an inner product, if $V$ is an abstract finite-dimensional real vector space, \emph{specifying an inner product on $V$ is equivalent to specifying a basis on $V$}.} Then, as $K$ is a bilinear form, there exists a matrix $\mathbf{K}$ where
\begin{equation}\label{kernel-history}
  K(x,y) = \langle x, \mathbf{K}y\rangle.
\end{equation}
The symmetry condition on the inner product $K$ forces the matrix $\mathbf{K}$ to be symmetric; we may appeal to the spectral theorem on symmetric matrices: $V$ has an orthonormal eigenbasis with respect to $\mathbf{K}$. Consider an eigenvalue $\mathbf{K} v = \lambda v$. Because $K$ is positive-definite, we know:
\[K(v,v) = \langle v, \mathbf{K}v\rangle = \lambda \langle v,v\rangle> 0.\]
Since the dot product of a vector with itself in any basis is nonnegative, this implies that the eigenvalue $\lambda$ is positive; so in its eigenbasis, $\mathbf{K}$ is diagonal with positive terms on its diagonal---it follows that $\mathbf{K}^{1/2}$ exists, and this allows us to write:
\begin{equation}
  K(x,y) = \langle \mathbf{K}^{1/2} x, \mathbf{K}^{1/2}y\rangle.
\end{equation}
We deduce that there exists a linear map $\Phi: V \to \bbR^n$ (spoiler: we call $\Phi$ a \emph{feature map}) where the inner product on $V$ directly corresponds to the usual dot product on $\bbR^n$. That is,
\begin{equation}\label{canonical-inner-prod}
  K(x,y) = \langle \Phi(x), \Phi(y)\rangle.
\end{equation}
In other (ridiculously abstract) words, Equation~\ref{canonical-inner-prod} precisely says:
\begin{proposition}\label{finite-IP}
  Let $V$ be an $n$-dimensional real vector space. If $K : V \times V \to \bbR$ is an inner product, then there exists a map $\Phi: V \to \bbR^n$ such that the following diagram commutes:
  \[\begin{tikzcd}
V \times V\ \  \arrow[dashed, rr, "\Phi \times \Phi"] \arrow[drr, "K"'] & & \bbR^n \times \bbR^n \ \arrow[d, "{\langle\cdot,\cdot \rangle}"] \\
& & \mathbb{R}
\end{tikzcd}
\]
  where $\langle \cdot, \cdot \rangle$ is the standard dot product on $\bbR^n$.
\end{proposition}
This is the main payoff to all this formality: we may view the pair $(\bbR^n, \langle\cdot,\cdot\rangle)$ as the canonical $n$-dimensional real inner product space. So, no matter which inner product space $(V,K)$ we want to study, we're in fact guaranteed the existence of an isomorphism $\Phi$ to the familiar inner product space $\bbR^n$, and we may freely interchange the two objects in our minds.

Furthermore, we are now justified in dropping the distinction between the formal dot product and inner product; let us denote both by $\langle \cdot, \cdot\rangle$.

\subsection{Generalization: Kernels and Hilbert Spaces}
First, we will need to extend the `similarity measure' from vector spaces $V$ to abstract sets $\mathcal{X}$. In the former, notice that given a basis $\{v_1,\dotsc, v_n\}$, the inner product $K$ is fully determined by the collection of values $K(v_i,v_j)$, where $i,j$ range between 1 and $n$. This follows from bilinearity of $K$ and is equivalent to saying that we can represent the bilinear form $K$ by a matrix of $n\times n$ values $\mathbf{K}$. In some sense, this means that there are only `$n$ ways' or `$n$ directions' in which objects of $V$ may be (dis)similar.

This suggests that we can generalize the definition of inner products because we didn't need all of $V$ to start with! Suppose someone secretly had an $n$-dimensional inner product space $(V,K)$, but just gave us $n$ linearly independent vectors, say $[n] := \{v_1,\dotsc, v_n\}$, along with the corresponding restriction of the similarity measure $K: [n] \times [n] \to \bbR$.

Our view of $[n]$ would just be a collection of $n$ abstract objects, and $K$ just an $n\times n$ matrix over $\bbR$ (this is often called the \emph{Gram matrix}). Still, we would have produced the same feature map $\Phi_{[n]}: [n] \hookrightarrow \bbR^n$, albeit restricted to $[n]\subset V$. But in the previous analysis, since $V$ and $\bbR^n$ are isomorphic via $\Phi$, this implies that $\Phi_{[n]}$ actually \emph{recovers} $V$ by its identification with $\bbR^n$; it is as though we've discovered the secret that $[n]$ lives inside the space $V$.

Very naturally, the generalization of $K$ from $[n]$ to $\mathcal{X}$ takes the view that $K$ is a similarity `matrix' on $\mathcal{X}$ (though we call it a \emph{kernel}):
\begin{definition} Let $\mathcal{X}$ be a nonempty set. A \emph{kernel}\footnote{Unfortunately, the terminology \emph{kernel} can be terribly confusing. Within functional analysis, kernels are often interchangably called \emph{kernel maps}, \emph{kernel functions}, and \emph{positive-definite kernels}. Yet, each of these also take on other meanings depending on author and context. `Kernels' and `kernel maps' are almost always synonymous. Sometimes, `kernel functions' denote kernels of the form $K(x,y) = k(x-y)$, in situations where subtraction is defined. And in many instances, when context is clear, the author cares only for positive-definite kernels, so omitting the specifier \emph{positive-definite}.

    Furthermore, \emph{kernels} used in this sense are unrelated to the algebraic kernel of a linear map. The terminology comes from the theory of integral operators. In particular, infinite-dimensional function spaces often have inner products of the form:
    \[\langle f, g\rangle = \int_{X \times X} K(x,y) f(x) g(y) \ \mathrm{d}x \mathrm{d}y,\]
    where the $K$ here performs the analogous role as $\langle x, \mathbf{K}y\rangle$ in Equation~\ref{kernel-history}. Actually, it performs precisely the role we desire in the general case. These maps $K$ were historically called kernels.
  } is a map $K : \mathcal{X} \times \mathcal{X} \to \bbR$. We say that $K$ is \emph{positive-definite} if for any finite subsets $A$ of $\mathcal{X}$, the corresponding Gram matrix of $A$ is symmetric and positive-definite.
\end{definition}

In the finite case, we produced an embedding $\Phi_{[n]} : [n] \to \bbR^n$ of $[n]$ satisfying Equation~\ref{canonical-inner-prod} ($K$ corresponds to inner product). As we had only $n$ objects, it is clear that we needed an inner product space with at most $n$ dimensions, hence $\bbR^n$.

However, in the general case, where $\mathcal{X}$ may be infinite, we might need infinite dimensions. Consider the real vector space $\bbR^\mathcal{X}$ (vector addition and scalar multiplication are defined pointwise), and the map $\Phi: \mathcal{X} \to \bbR^\mathcal{X}$ defined by:
\[\Phi(x) :=K(\cdot, x).\]
For short, let $k_x := \Phi(x)$. Then, $\Phi$ maps $\mathcal{X}$ into a subset of $\bbR^\mathcal{X}$; let $V = \mathrm{span}(\Phi(\mathcal{X}))$ be a linear subspace of $\bbR^\mathcal{X}$, so that elements $f$ of $V$ are of the form:
\[f= \alpha_1k_{x_1} + \dotsm + \alpha_n k_{x_n},\]
where $\alpha_i \in \bbR$ and $n$ ranges over $\bbN$. We claim that because $K$ is a positive-definite kernel, there exists a well-defined bilinear map on $V$, corresponding to our desired inner product. Of course, we want the inner product to satisfy:
\[\langle k_x, k_y\rangle = K(x,y).\]
But in fact, once we specify this, linearity determines the inner product on general elements $f = \sum_{i=1}^{n_1} \alpha_i k_{x_i}$ and $g = \sum_{j=1}^{n_2} \beta_j k_{y_j}$:
\[\langle f, g \rangle = \sum_{i=1}^{n_1} \sum_{j=1}^{n_2} \alpha_i \beta_j K(x_i, y_j).\]
This should feel quite familiar, recalling how the Gram matrix of $n$ linearly independent vectors fully determined the inner product in the finite case. However, we must be a little more careful here, as we are not guaranteed that the collection of $k_x$'s are linearly independent. In particular, suppose that $f$, as follows, is identically zero:
\[0 \equiv f = \sum_{i=1}^n \alpha_i k_{x_i}.\]
That is, $f(x) = 0$ for all $x \in \mathcal{X}$. Then, $\langle f, g\rangle$ better equal 0 for all $g \in V$. Notice that it is sufficient to show that $\langle f, k_x\rangle = \langle k_x , f\rangle = 0$ for each $x \in \mathcal{X}$. And indeed, by assumption
\[\langle k_x, f\rangle = \sum_{i=1}^n k_{x_i}(x) = f(x) = 0.\]
This proves that $V$ admits an inner product that is compatible with $K$ in the sense of Equation~\ref{canonical-inner-prod}. We will in fact go beyond producing an inner product space $V$; we can complete the space with respect to the norm induced by the inner product, producing a \emph{Hilbert space}.

That is, let $\mathcal{H}$ be the completion of $V$ by taking equivalence classes of Cauchy sequences on $V$. We claim that $\mathcal{H} \subset \bbR^\mathcal{X}$ is a collection of functions in the sense that $f(x) < \infty$ for all $f \in \mathcal{H}$ and $x \in \mathcal{X}$.\footnote{Note that in general, it is not the case that the completion of a function space consists of functions. Elements of $L^2(\bbR)$, for example, are equivalence classes of functions.} But let us relegate the proof to references, say \cite[Thm 3.16]{P2009}, for it is not particularly enlightening. However, the consequences are quite important.

First of all, we immediately attain the analog to Proposition~\ref{finite-IP}:

\begin{proposition}[Moore-Aronszajn]
  Let $\mathcal{X}$ be a set, and $K: \mathcal{X} \times \mathcal{X} \to \bbR$ a positive-definite kernel. Then, there exists a map $\Phi: \mathcal{X} \to \mathcal{H}$ into a Hilbert space such that the following diagram commutes:
    \[\begin{tikzcd}
\mathcal{X} \times \mathcal{X}\ \  \arrow[dashed, rr, "\Phi \times \Phi"] \arrow[drr, "K"'] & & \mathcal{H}\times \mathcal{H} \ \arrow[d, "{\langle\cdot,\cdot \rangle_\mathcal{H}}"] \\
& & \bbR
\end{tikzcd}
    \]
    where $\langle \cdot, \cdot\rangle_\mathcal{H}$ is the inner product on $\mathcal{H}$.
\end{proposition}
This is a powerful result because we may now view $\mathcal{X}$ not as an abstract set, but as vectors within a Hilbert space, which carries with it both algebraic and geometric properties that $\mathcal{X}$ now inherits. It is particularly amazing because the only thing we needed required was a positive-definite kernel $K$ to obtain $\mathcal{H}$. And in fact:
\begin{fact}
  The Hilbert space $\mathcal{H}$ induced by $K$ is unique. We call $\mathcal{H}$ the \emph{reproducing kernel Hilbert space (RKHS)} on $\mathcal{X}$ with respect to $K$. (See \cite[Prop 3.3]{P2009}).
\end{fact}

Later, we'll see a characterization: if $\mathcal{H}$ is an RKHS, then $\mathcal{H}$ induces a unique kernel $K$ on $\mathcal{X}$. So, analogous to the finite-dimensional case, we can think of $\Phi(\mathcal{X})$ in $\mathcal{H}$ as the `canonical form' of $\mathcal{X}$ with respect to the similary measure $K$.



Despite this, the way we've presented the embedding of $\mathcal{X}$ into $\mathcal{H}$ is somewhat unnatural. While it makes sense to consider the collection of functions $k_x$, the identification of $x$ with $k_x$ may seem \emph{ad hoc}. But if we venture a bit further into functional analysis, we may obtain a clearer understanding of RKHS's. Along the way, hopefully we can clarify why we might call this identification a feature map.

\subsection{Feature Maps}
Let us once again consider an abstract set $\mathcal{X}$, and a collection of functions over $\mathcal{X}$, say $\mathcal{H} \subset \bbR^\mathcal{X}$. We may imagine each $f \in \mathcal{H}$ as an observable or a feature, where $f(x)$ gives the \emph{measured value} of the object $x$ by feature $f$.

As a result, every $x \in \mathcal{X}$ is associated to the Cartesian product of measurements,
\begin{equation}\label{dual}
  x \overset{\Psi}{\mapsto}\prod_{f \in \mathcal{H}} f(x).
\end{equation}
The high-level punchline here will be that $\Psi(x)$ is actually a point in the \emph{dual} of $\mathcal{H}$, following the canonical mapping of $\mathcal{X}$ into the double dual $\mathcal{X}^{**}$. But in the case where $\mathcal{H}$ is a Hilbert space and all the $\Psi(x) \in \mathcal{H}^*$ are bounded (i.e. continuous), then $\mathcal{H}$ is isomorphic to $\mathcal{H}^*$. It turns out that $\Psi(x)$ is naturally identified with $k_x$, justifying the initial definition $\Phi : x \mapsto k_x$.

Before elaborating on this, let us provide a visual example; perhaps a bit of concreteness will convince the reader why we should even care about the association in Equation~\ref{dual} in the first place.

Imagine $\mathcal{X}$ is a set of abstract cats, while $\mathcal{H} = \{f_1,\dotsc, f_n\}$ represents the pixels within the lens of a camera. Given a cat $x$, the value $f_i(x)$ gives the color the $i$th pixel on the camera. Then, the product $\Psi(x)$ in Equation~\ref{dual} represents the \emph{image} of the cat---the collection of colors all the camera pixels see as the camera takes the picture of cat $x$.

Here, it makes sense to call the mapping in Equation~\ref{dual} the \emph{feature map}, for it maps the cat to a collection of features, specifically its colors at different locations. In general, given any function $f : \mathcal{X} \to \bbR$, we may dually view points of $x$ as functionals, $\mathrm{ev}_x : \mathcal{H} \to \bbR$, where
\[\mathrm{ev}_x(f) := f(x).\]
In functional analysis, we say that $\mathrm{ev}_x$ is the \emph{evaluation at $x$}, and if it is continuous, then we view it as an element of a larger set of continuous functionals on $\mathcal{H}$, the \emph{continuous dual}, $\mathcal{H}^*$.\footnote{The earlier point that $\mathcal{H}$ is a collection of functions over $\mathcal{X}$ is equivalent to saying that the evaluation functionals are continuous.} Let us return to kernels. But whereas in the previous section, the feature map fell out of the construction almost incidentally, here, we begin with it: $\Psi(x) = \mathrm{ev}_x$.

In particular, we equivalently define an RKHS (equivalency will become clear):
\begin{definition}
Let $\mathcal{H} \subset \bbR^{\mathcal{X}}$ be a Hilbert space. It is an RKHS if for all $x \in \mathcal{X}$, the evaluation functional $\mathrm{ev}_x$ is continuous.
\end{definition}
That is, $\mathcal{H}$ is an RKHS if $\Psi(\mathcal{X}) \subset \mathcal{H}^*$. As made intuitive by the cat images example, $\Psi(\mathcal{X})$ is the representation of $\mathcal{X}$ seen through the lens of features $\mathcal{H}$. And if those features are quite discerning, for any distinct objects $x,y \in \mathcal{X}$, we might expect that their differences will be measurable; there exists some $f \in \mathcal{H}$ such that $f(x) \ne f(y)$. When this is the case, we say that $\mathcal{H}$ \emph{separates points}, and it follows that:
\begin{proposition}
  The mapping $\Psi : \mathcal{X} \to \mathcal{H}^*$ is injective if and only if $\mathcal{H}$ separates points.
\end{proposition}

So in many cases, we may think of $\Psi$ as an embedding of $\mathcal{X}$ into $\mathcal{H}^*$. But then, Riesz representation theorem states that $\mathcal{H} \cong \mathcal{H}^*$. This is a powerful isomorphism, for it implies that every $x$, which corresponds to a unique $\Psi(x) \in \mathcal{H}^*$, then corresponds to some unique point $k_x$ in $\mathcal{H}$ such that:
\[\Psi(x) = \langle \cdot, k_x\rangle_\mathcal{H}.\]
We call $k_x$ the \emph{reproducing kernel for the point $x$}, and for consistency, we can let $\Phi: x \mapsto k_x$. But via the isomorphism, we may afford to drop the distinction between $\Psi$ and $\Phi$.

Now, as vectors in a Hilbert space, the points $k_x$ inherits the similarity measure from the inner product on $\mathcal{H}$, producing a positive-definite kernel on $\cX$:
\[K(x,y) = \langle k_y, k_x\rangle_\mathcal{H}.\]
We call $K$ the \emph{reproducing kernel of $\mathcal{H}$}, for $\mathcal{H}$ uniquely determines $K$. Recalling from before---every kernel $K$ uniquely determines an $\mathcal{H}$---we now see that these two views of RKHS's we've provided are equivalent. And whenever we are given a set $\mathcal{X}$ with a positive-definite kernel, we can reason as though we know that they belong to a Hilbert space $\mathcal{H}$.

\subsection{In Practice}
In many cases, one important limitation is the infeasibility of constructing $\Phi$. And especially in a computational setting, if $\mathcal{H}$ is infinite dimensions, representation of its vectors becomes even more problematic.

Still, we can still take advantage of the existence of $\Phi$; sometimes we can reason about properties of $\mathcal{X}$ as points in $\mathcal{H}$, but pull the analysis back through $K$.

In the rest of the paper, we'll look at how this may be done for LSH.

\section{Similarity Metrics, Unusual and Usual} % dspeyer 
  
LSH generally applies to distance metrics, hence the inequalities
discussing $||x_1-x_2||$.  Kernels generally apply to similarity
metrics.  How do we combine these ideas?

The simplest case is when the data is already on a sphere in $bbR^n$
with $\ell_2$ distance.  In this case, $||x_1-x_2||=\sqrt{\langle
  x_1-x_2, x_1-x_2 \rangle}=\sqrt{\langle x_1,x_1 \rangle+\langle
  x_2,x_2 \rangle - 2 \langle x_1,x_2 \rangle} = \sqrt{2(1-\langle
  x_1,x_2 \rangle)}$.  In this case, saying $||x_1-x_2||<r$ as we did
in our original inequality is equivalent to saying $\langle x_1,x_2
\rangle > 1-r^2/2$.  This means the $c$ term becomes squared, which
works to our benefit when constructing an Approximate Nearest Neighbor
algorithm.

Not every distance metric can be converted into a similarity metric as
cleanly.  If $\ell_1$ distance could be converted into a similarity
metric and thence to an $\ell_2$ distance without sacrificing cr-near
properties, such a transform could be used to perform cr-near neighbor
searches with $\rho$ comparable to $\ell_2$ cr-nn, but this has been
proven impossible [TODO: cite].

One can convert a distance to a similarity using general formulas such
as $k(x,y)=1-d(x,y)^2/\max(d)^2$ (analogous to the unit sphere) or
$k(x,y)=\exp(-d(x,y))$ (if we wish to operate elegantly without
needing to find the maximum distance), but these are not guaranteed to
have nice cr-nn properties.

Often it is better to construct a new metric.  For example, the usual
distance metric in genetic sequences is string-edit distance, usually
with indels costing twice as much as changes.  This is too similar to
$\ell_1$ to likely permit a good conversion.  However, one can also
count k-mer frequency and take a normalized inner product (or, for
longer k-mers, Hamming distance subtracted from $4^k$).  These are
proper similarity metrics which predict evolutionary homology and
analogy of function almost as well as string edit distance [TODO:
  cite].  The true test of similarity metric quality is an empirical
one.

Similarly in images, you can describe distance as the length of the
series of edits to transform one image into the other (where each step
is artistically simple), but this is often not the best predictor of
human-judged similarity.  A more popular technique is to extract a set
of features by convolution with random Gaussian or sinusoidal filters
and use some form of normalized inner product.

For a more general solution, one could train a deep neural net on
pairs of objects that are and are not ``essentially'' the same.  The
final output would be a probability that they are.  A logistic forcing
function would guarantee positive-definiteness, and adding all pairs
of training data in both directions would guarantee symmetry.

In general, any similarity metric that seems intuitively correct or
that correlates well to any practical purpose is likely to be
positive-definite and symmetric, and therefore a valid kernel.

\section{KLSH}

Now we are ready to put the pieces together into KLSH.  Suppose we
have a set of objects $\cX$ on which the only defined function is a
simularity metric $K:\cX \times \cX \rightarrow [0,1]$ (with
$K(x,x)=1$).  In the most
common example, $\cX$ is a collection of images and $K$ is whatever
kernel the computer
vision community recommends.

Note that $\cX$ is \emph{not} generally a vector space.  We cannot add two
images, nor take their inner product.  We cannot generate a random
image drawn from a Gaussian distribution.  These things are not
defined.  Furthermore, we might be unable to find a function $\Phi$ that would map
the image into a vector space.  We know one exists, but we do not have
it, nor do we know any interesting properties of the Hilbert space it
maps into.

Nevertheless, we would like to apply LSH techniques that are defined
in the kernel space to search problems in the object space.

Specifically, we will start by applying LSH of the unit sphere (which
is where we are, by the $K(x,x)=1$ definition earlier) by hyperplane
carving.

\subsection{Hyperplane Carving}

The simplest way to do LSH on a unit sphere in $\bbR^n$ with inner
product similarity is to carve the sphere in half with a random
hyperplane.  This is equivalent to taking a vector perpendicular to
that hyperplane (traditionally called $g$) and taking the sign of its
inner product with the vector in question.  The vector $g$ must be
drawn at random from a spherical Gaussian distribution.  This means
that regardless of $x_1$ and $x_2$, whether $h(x_1)=h(x_2)$  is
determined by whether $g$ falls into the region which makes it so,
where the size of that region is determined by $\langle x_1,x_2
\rangle$.


%% \subsection{Normal Distributions in RKHS} % dspeyer

%% The hyperplane carving technique requires drawing points $g$ from a
%% normal distribution.  As we observed, we cannot do this in $\cX$
%% because it is undefined, and we cannot do this in $\cH$ because we
%% cannot describe any point in $\cH$.

%% Nevertheless, we can draw points in $\cH$ from a normal distribution,
%% and compute their inner products with points of the form $\Phi(x)$.

%% The key is the central limit theorem.  This theorem states that if you
%% draw $n$ points i.i.d. from any distribution and average
%% them, the average will approximately be drawn from a normal
%% distribution, whose mean is the same as that of the original, and
%% whose covariance is that of the original times $\sqrt{n}$.  The
%% ``approximately'' goes away as $n$ approaches infinity.  In general,
%% $n=30$ is sufficient to make a good enough approximation, though the



\subsection{Attempt at KLSH} % geelon
Let's try to kernelize the ball-carving LSH method. Given $\mathcal{X}$ and a
normalized positive-definite kernel $K$,\footnote{One can normalize any positive-definite
  kernel $K(x,y)$ by:
  \[K'(x,y) = \frac{K(x,y)}{\sqrt{K(x,x)K(y,y)}}\].}
one very natural way is to consider the embedding $\Phi(\mathcal{X}) \subset \mathcal{H}$
inside the RKHS.

In particular, because $K$ is normalized, all the points $\Phi(x)$ fall onto the unit sphere
in $\mathcal{H}$, so it seems that we just need to ball-carve the sphere in $\mathcal{H}$.
In the usual LSH situation in $\bbR^n$, we drew a Gaussian $g \sim \cN(0,1)^n$ and hashed
according the function:
\[h(x) = \sign\ \langle x, g\rangle.\]
So it seems that we just need to replicate this within $\mathcal{H}$ instead of $\bbR^n$:
\[h(x) = \sign\ \langle \Phi(x), g\rangle_\mathcal{H},\]
where $g$ is drawn according to a Gaussian in $\mathcal{H}$. But here, we run into two problems:
\begin{enumerate}
\item if $\mathcal{H}$ is infinite-dimensional, it does not have a canonical Gaussian distribution, 
\item if we don't have $\Phi$, how can we compute the inner product of $\Phi(x)$ with $g$?
\end{enumerate}

Let's take a closer look at why there is no canonical Gaussian distribution---in fact, this will provide us with some heuristic to think about how $\Phi$ might embed $\cX$ into $\cH$. From that insight, we will be able to reduce the infinite-dimensional problem down to a finite-dimensional one, which we essentially already know how to solve.

\subsection{Projection to Finite Dimensions}

\begin{proposition}[{\cite[Example 2.2]{L2012}}] \label{infinite-distribution}Let $\mathcal{H}$ be a separable Hilbert space.
  A Gaussian distribution with covariance $\Sigma$ exists iff, in an appropriate base, $\Sigma$
  has a diagonal form with non-negative eigenvalues, and the sums of these eigenvalues is finite.
\end{proposition}

In particular, this implies that no analog of the normal distribution $\mathcal{N}(0,I)$ exists in $\cH$ as the trace of $I$ is infinite. Although we won't prove this,\footnote{Additionally, the reader might notice that we have not actually defined what a Gaussian measure on $\cH$ is; we refer the reader to \cite{L2012} for details.} we can give intuition: the normal distribution in $\bbR^n$ is the product of $n$ independent normal Gaussians along $n$ orthogonal directions. As the number of dimensions go to infinity, we would find that the probability distribution would converges to 0 on $\mathcal{H}$. And so any Gaussian that exists on $\mathcal{H}$ must have `most' of its variance concentrated within a finite number of directions.

But this leads to some possible intuition utilized by papers such as \cite{K2012,J2015}, ultimately resulting in the KLSH method. We might ask how is $\Phi(\cX)$ `distributed' within $\cH$?\footnote{This question is not very precise: do we have a measure on $\cX$? Does that induce a measure in $\cH$? Or, is there already a measure on $\cH$, and we assume that $\Phi(\cX)$ is measurable? It might be worth pursing these questions more in the future.} The proposition above therefore suggests that most of the variance of $\Phi(\cX)$ lie on a finite number of dimensions.

Slightly more precisely we could look at how a large but finite subset $S \subset \cX$ is distributed within $\cH$ (so as to avoid unreliable intuition if $\cX$ is infinite). Let $\Sigma$ be the covariance of $\Phi(S) = \{\Phi(x_1),\dotsc, \Phi(x_m)\}$. For simplicity, let's consider $\Sigma$ in its eigenbasis, so that
\[\Sigma = \sum_{i=1}^{d_\Phi} \lambda_i v_i \otimes v_i,\]
where $\lambda_1 \geq \lambda_2 \geq \dotsc \geq 0$ are the eigenvalues and $v_1,v_2,\dotsc$ are the (orthonormal) eigenbasis (by $d_\Phi$, we mean the dimension of $\cH$, which may be infinite). By Proposition~\ref{infinite-distribution}, the sum of the eigenvalues is bounded, so for any $\epsilon > 0$, there is some $N \in \bbN$ such that:
\[\sum_{i = N}^{d_{\Phi}} \lambda_i < \epsilon.\]
Thus, if we project $\Phi(x)$ onto the first $N$ eigenbasis (so, truncate $\Phi(x)$ at $N$ coordinates in the covariance eigenbasis):
\[\pi: \Phi(x) \mapsto \sum_{i=1}^N \langle \Phi(x), v_i\rangle \ v_i, \]
then the expected truncation error will be less than $\epsilon$. This essentially describes \emph{principal component analysis} from within the RKHS (this is called KPCA).

If we project $\cH$ into a finite-dimensional vector space by this truncation, then we can follow up with the usual LSH algorithm. At its core, this is how KLSH works, as described in \cite{K2012}. Interestingly enough, while they did not at first view KLSH as a combination of KPCA and standard LSH at first, their later paper \cite{J2015} proved the equivalence of their original method to KPCA+LSH. While the intuition is useful, we still need to resolve one more matter: truncated or not, how is $\langle \Phi(x), g\rangle$ computed? We still only have access to $\cH$ through the kernel map $K$. It turns out that this will enough.

\subsection{Using the Kernel}

[NEED TO WRITE]

That is, suppose we had some random variable $X \in \cX$, inducing a random variable $\Phi(X) \in \cH$, with some mean $\mu$ and covariance $\Sigma$. Then, if we drew $t$ samples, $\Phi(X_1),\dotsc, \Phi(X_t)\}$, CLT implies that the random variable
\[\Sigma^{-1/2} Z := \Sigma^{-1/2} \left(\sqrt{t} \sum_{i=1}^t \Phi(X_i) - \mu\right)\]
will converge to a standard Gaussian as $t$ approaches $\infty$.





In general,
$n=30$ will be sufficient in KLSH to make a good enough approximation, though the
theoretical bound is poorly studied, and extremely multimodal
distributions might require larger values of $n$.




Let $z$ be a realization of $Z$. Then, it follows that we could ball-carve using the hash:
\[h(x) = \sign \ \left\langle \Phi(x), \Sigma^{-1/2}z\right\rangle.\]
Let's examine the term $\left\langle \Phi(x), \Sigma^{-1/2}z\right\rangle$ in the eigenbasis of $\Sigma$, where $\Sigma$ has eigenvalues $\lambda_1 \geq \lambda_2\geq \dotsc \geq 0$ and eigenvalues $v_1,v_2,\dotsc$. Then, 






If we select a random point $x$ from our database (equal chance of
each point) and consider its $\Phi(x)$, this is a distribution in
$\cH$, albeit a strange and discrete one.  That's good enough for the
central limit theorem to apply.  This lets us ``draw'' a point in
$\cH$ from $\cN(\mu,\Sigma)$.  Furthermore, we can convert this into a
point from $\cN(0,I)$ by subtracting $\mu$ and multiplying by
$\Sigma^{-1/2}$.  That is to say:

\begin{equation*}
  g = \Sigma^{-1/2}\left(\frac{1}{n}\sum_i^n \Phi(x_i) \right) - \mu  \sim \cN(0,I)
\end{equation*}

Granted, we still cannot compute this.  By approximating $\Sigma$,
however, we can compute per-point scalars $w_i$ such that $g=\sum_i
w_i\Phi(x_i)$.  We can then use this form to compute
inner products:

\begin{eqnarray*}
  \langle \Phi(q), g \rangle
  & = & \langle \Phi(q), \sum_i w_i\Phi(x_i) \rangle \\
  & = & \sum_i w_i \langle \Phi(q), \Phi(x_i) \rangle \\
  & = & \sum_i w_i K(q, x_i) \\
\end{eqnarray*}

%% We can then take an eigen decomposition of $\Sigma^{-1/2}$ into vectors $v_j$ and values $\lambda_j$ and use this basis to describe $\Phi(x)=\sum_j \langle x,v_j \rangle v_j$.  Once we've described the average in the eigenbasis, we can regard the covariance multiplication as applying the eigenvalues.  From there, it's just a matter of changing the order of summation:

%% \begin{eqnarray*}
%%   \Sigma^{-1/2}\left(\frac{1}{n}\sum_i^n \Phi(x_i) \right)
%%   & = & \Sigma^{-1/2}\frac{1}{n}\sum_i^n \sum_j \langle \Phi(x_i),v_j \rangle v_j \\
%%   & = & \frac{1}{n} \Sigma^{-1/2} \sum_j \left( \sum_i^n \langle \Phi(x_i),v_j \rangle \right ) v_j \\
%%   & = & \frac{1}{n} \sum_j \lambda_j \left( \sum_i^n \langle \Phi(x_i),v_j \rangle \right ) v_j \\

\subsection{Analysis of Algorithm}

\section{Data-Dependent KLSH} % dspeyer

\subsection{Data-Dependent LSH}

Random hyperplanes are not the most efficient form of LSH for a unit
sphere.  They achieve only [TODO: look this up] while the best-known
technique achieves $\rho=1/(2c^2-1)$.  This is better than the
theoretical limit and is made possible by basing the hashing functions
on the data.

The first step is to replace the hyperplane-carving hash function with
a cap-carving one.  Instead of checking the sign of $\langle g, q
\rangle$, we test if it is greater than $d^{1/4}$.  And
instead of simply reporting ``no'' if not, we keep picking $g$s
according to some deterministic psuedorandom rule until we find one
that is.

If the data is spread evenly over the sphere, then we expect a very
small fraction of points in each cap.  To speak somewhat imprecisely,
the inner product is the sum of the product in each co-ordinate.  Each
of those products has expected value zero (by symmetry) and a very
small variance (because of the cap on the total length).  Summing many
of them drives the variance down further and allows us to apply a
Chernoff bound.

The low fraction of points means that if a query and a database point
are in the same cap, it's probably because they're close to one
another.  Note that we are assuming the data is sparse compared to
the possibility space.

This allows us to get $\rho=1/(2c^2-1)$, but only if the data is
random or close to random.

\subsubsection{Approximate Evenness}

What we mean by ``close to random'' is that no cap exceeds its
expected number of points by more than a constant factor.

This is not something we are likely to have.  First, a $[0,1]$-ranged
similarity function will map all points into the ``upper right''
orthant.  More worryingly, we can expect any real-world data to be
extremely clumpy.

It is possible to extract a reasonable number of ``lumps'' and leave
an approximately even residue by taking a small number of test points
from the dataset, computing their similarities to all other points and
noticing if any are close to many.  This is a slightly superlinear
time preperatory step that tells us about troublesome balls centered
on known datapoints.

The standard technique is then to remove those balls and process them
separately, by ignoring their sphere-segment structure and slicing
them into new spheres.  While complex, the operation is
time-efficient.  Unfortunately, the process of sphere-slicing a ball
likely would not be possible in kernel space.

\subsection{Smaller Caps}

What might be possible would be to first carve the dense regions with
smaller caps, then the rest of the ball with regular size ones.  If we
increase the threshold on $\langle g,q \rangle$ to decrease the
diameter of the cap proportionally to the diameter of the ball, this
should get us something close to the correct number of points.  

It may still be necessary to recurse on dense regions inside dense
regions, but no more so than in the original algorithm.

Unlike sphere carving, this should be possible to implement in kernel
space.  The points can be drawn by taking the center (a point in the
database!), adding a normal vector resized to the ball's radius, and
normalizing the result.  These operations are straightforward to
reduce to computable kernel operations.

We don't have space to fully develop this idea, which we have neither
proved nor tested empirically.  But, from everything we can see, there
have been no attempts to combine kernelized and data-dependent LSH, so
we are optimistic about this approach.

\section{References}
\beginrefs

\bibentry{A1950} Aronszajn, Nachman. \emph{Theory of reproducing kernels}. Transactions of the American mathematical society 68.3 (1950): 337-404.

\bibentry{E2016} Eldredge, Nathaniel \emph{Analysis and probability on infinite-dimensional spaces}. arXiv preprint arXiv:1607.03591 (2016).

\bibentry{G2013} Gretton, Arthur. \emph{Introduction to RKHS, and some simple kernel algorithms}. Adv. Top. Mach. Learn. Lecture Conducted from University College London (2013).

\bibentry{H2008} Hofmann, Thomas, Bernhard Schölkopf, and Alexander J. Smola. \emph{Kernel methods in machine learning}. The annals of statistics (2008): 1171-1220.

\bibentry{J2015} Jiang, Ke, Qichao Que, and Brian Kulis. \emph{Revisiting kernelized locality-sensitive hashing for improved large-scale image retrieval}. Proceedings of the IEEE Conference on Computer Vision and Pattern Recognition. 2015.
  
\bibentry{K2012} Kulis, Brian, and Kristen Grauman. \emph{Kernelized locality-sensitive hashing}. IEEE Transactions on Pattern Analysis and Machine Intelligence 34.6 (2012): 1092-1104.

\bibentry{L2012} Lifshits, Mikhail. \emph{Lectures on Gaussian processes}. Lectures on Gaussian Processes. Springer Berlin Heidelberg, 2012. 1-117.

\bibentry{P2009} Paulsen, Vern I., and Mrinal Raghupathi. \emph{An introduction to the theory of reproducing kernel Hilbert spaces}. Vol. 152. Cambridge University Press, 2016. 


\endrefs
%\end{multicols}


\end{document}
